{{REWRITTEN_CODE}}
\documentclass[11pt]{article}
\usepackage{amsmath, amssymb, geometry, graphicx, xcolor, hyperref}
\geometry{margin=1in}
\title{Advancing Cognitive-Inspired AI Integration:\ A Multilayer Neuro-Symbolic Framework for Clinical and Educational Reasoning}
\author{Your Research Team}
\date{July 2025}

\begin{document}
\maketitle

\begin{abstract}
This whitepaper introduces a three-phase roadmap for implementing a cognitively inspired AI system based on multilayer network theory, multi-agent reinforcement learning, and neuro-symbolic integration. We detail the integration of \textsc{Lexical Viability Component} (\textsc{lvc}) scoring, \textit{Retrieval-Augmented Generation} (\textsc{rag}), and competitive debate mechanisms to enhance reasoning accuracy, mitigate bias, and support educational and clinical deployment. The proposed framework demonstrates \textbf{technical feasibility}, \textbf{cognitive plausibility}, and \textbf{real-world applicability} across mathematical reasoning, cognitive assessment, and personalized learning.
\end{abstract}

\section{Introduction}
\label{sec:intro}
Cognitive-inspired artificial intelligence seeks to bridge the gap between \textit{statistical pattern recognition} and \textit{human-level understanding}. By integrating principles of neural-symbolic reasoning, attentional decoupling, and multi-agent discourse, we propose a framework that emulates foundational cognitive processes. Our system addresses critical limitations in current AI models, particularly in \underline{explainability}, \underline{bias mitigation}, and \underline{authentic mathematical reasoning}.

\section{Phase 1: LVC Integration and Cognitive Grounding}
\label{sec:phase1}
\subsection{LVC Scoring and Semantic Kernels}
The \textsc{Lexical Viability Component} (\textsc{lvc}) identifies words and concepts most central to cognitive fluency. Empirical studies show these terms enhance \textbf{nameability}, \textbf{recall}, and \textbf{semantic access} in clinical and healthy populations.

\textbf{Integration:} We compute \textsc{lvc}-based scores to define $R_{\text{cognitive}}$ in our cognitive plausibility function:
$$\Psi(x) = f\left(x, R_{\text{cognitive}}, \alpha(t)\right)$$

\subsection{Multilayer Attention Mechanisms}
We use recursive multi-scale attention across \textit{semantic}, \textit{syntactic}, and \textit{phonological} layers to decouple recognition from attention focus. This improves symbolic grounding and supports both focused and divergent thinking patterns.

\subsection{Mathematical Cognition Parsing}
Inspired by neuroimaging findings, we develop \textsc{lvc}-aware parsers that tokenize and reason over mathematical expressions with \texttt{sub-200ms} latency, approximating human visual cognition.

\section{Phase 2: Multi-Agent Debate and Adaptive Reasoning}
\label{sec:phase2}
\subsection{Competitive Debate Mechanisms}
Multiple agents, each embodying different bias priors ($\beta$), participate in \textit{dialectical inference}:
$$P(H \mid E, \beta) = \frac{P(E \mid H, \beta) \cdot P(H \mid \beta)}{P(E \mid \beta)}$$
A \textbf{moderator agent} aggregates evidence-based beliefs to minimize unjustified assumptions.

\subsection{RAG-Enhanced Semantic Understanding}
\textit{Retrieval-Augmented Generation} (\textsc{rag}) modules provide grounded, current, and context-relevant inputs. This enhances the evidence $E$ available to each reasoning agent.

\subsection{Reinforcement-Based Bias Calibration}
Agents dynamically update $\beta$ via reinforcement learning to optimize long-term reasoning accuracy:
$$\max_{\beta} \; \mathbb{E}\left[R(H, \beta)\right]$$

\section{Phase 3: Clinical and Educational Deployment}
\label{sec:phase3}
\subsection{Educational Technology}
\begin{enumerate}
\item \textbf{Adaptive tutors} track cognitive state via attention patterns
\item \textbf{Knowledge graphs} driven by \textsc{lvc} scores personalize content
\item \textbf{Symbolic-neural integration} supports multi-modal reasoning
\end{enumerate}

\subsection{Clinical Application via CARE Framework}
Deployment adheres to the \textsc{Clinical AI Readiness Evaluator} (\textsc{care}) workstreams:
\begin{enumerate}
\item Regulatory compliance
\item Data pipeline validation
\item Cognitive domain modeling via \textit{Harmonized Cognitive Assessment Protocol} (\textsc{hcap})
\end{enumerate}

\section{Validation Case: Mathematical Expression Processing}
\label{sec:validation}
\subsection{Problem:} Solve $2x + 3 = 7$
\subsection{Cognitive Metrics:}
\begin{itemize}
\item $\alpha(t) = 0.6$ (\textit{Balanced neural-symbolic})
\item $R_{\text{cognitive}} = 0.12$ (\textit{Low cognitive penalty})
\item $P(H \mid E, \beta) = 0.83$ (\textit{High bias-adjusted confidence})
\end{itemize}

\textbf{Solution Process:}
\begin{align}
2x + 3 &= 7 \\
2x &= 7 - 3 \\
2x &= 4 \\
x &= \frac{4}{2} = 2
\end{align}

\section{Strategic Outlook}
\label{sec:outlook}
\subsection{Opportunities}
\begin{itemize}
\item Extension to \textbf{scientific discovery} and \textbf{decision-making}
\item \textit{Federated learning} for privacy-preserving adaptation
\item \underline{Consciousness modeling} for altered cognitive states
\end{itemize}

\subsection{Challenges}
\begin{itemize}
\item Computational scalability (\texttt{multi-agent simulation})
\item Modular validation complexity
\item Regulatory approval timelines
\end{itemize}

\section{Conclusion}
\label{sec:conclusion}
This whitepaper proposes a novel architecture for cognitive AI grounded in \textit{multilayer network theory} and \textit{symbolic-neural integration}. By structuring implementation across \textbf{three strategic phases}, we enable scalable, explainable, and human-aligned AI capable of transforming educational and clinical reasoning. Future work will focus on empirical validation and cross-domain generalization as outlined in Section~\ref{sec:validation}.

\end{document}
